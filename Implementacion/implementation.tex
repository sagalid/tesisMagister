\subsection{Implementation architecture}
The implementation of the algorithm was developed using the Python programming language which allows work quickly and integrate systems more effectively.
As data persistence, a MySql relational database was used. All of the above was mounted on an instance of the free Amazon layer, specifically EC2 - Linux.
%------------------------
Como sistema de control de c�digo fuente se utiliz� GIT, el cual permiti� mantener un control de c�digo fuente local tanto para el algoritmo desarrollado como para la documentaci�n asociada a esta tesis. Se realizaron constantes r�plicas contra un repositorio GIT en nube, espec�ficamente GITHUB.
 
\subsection{Binary Global-Best Harmony Search Metaheuristic}
Los par�metros que se presentan a continuaci�n fueron fruto de pruebas exhaustivas y multiples ejecuciones que permitieron probar m�s all� de los componentes aleatorios un comportamiento correcto para la metaheur�stica, es decir presentando un grado de efectividad m�s alto comparado con otras metaheur�sticas.
 
El proceso comienza con la generaci�n de una poblaci�n inicial de armon�as que son vectores compuestos por digitos binarios $d$, en un repetici�n del proceso que se denomina improvisaci�n. La cantidad de improvisaciones se denota por $NI$. Finalizadas las improvisaciones, se cuenta con una cantidad de armon�as sobre las cuales se ejecuta un proceso de reparaci�n con dos fases de acuerdo al Agoritmo \ref{alg:addAndDrop}. Una vez que las armon�as se encuentra reparadas y cumplen con las restricciones definidas en la Matriz $A$ del SCP, se almacena la mejor y la peor armon�a, denotadas por $x_{best}$ y $x_{worst}$ respectivamente. El detalle del comportamiento de la metaheuristica puede ser revisado en el diagrama de flujo \ref{dia:FlowBGBHS}.
 
 Durante las pruebas se detect� al momento de generaci�n de las armon�as que un par�metro variable $p$ en los ensayos de Bernoulli, permit�a obtener una explotaci�n en la �rea de b�squeda logrando escapar de �ptimos locales. La variaci�n del par�metro $p$ de los ensayos de Bernoulli fue ajustado de acuerdo  a la funci�n que se presenta a continuaci�n.
 
 %%Insertar funci�n para p de Bernoulli
 %%%%%%%%%%%%%%%%%%%%
 
 Se realiz� un an�lisis de resultados, comparando los valores de multiples ejecuciones de la metaheur�stica con un par�metro $p$ fijo, versus un par�metro $p$ 
 
 
 
 %------------------------

The HS is good at identifying the high performance regions of the solution space in a reasonable time, but poor at performing local search \cite{DBLP:journals/eswa/XiangALHZ14}. Namely, there is an unbalance between the exploration and the exploitation of HS. Furthermore, HS designed for continuous space cannot be directly used to solve discrete combinatorial optimization problems.

In order to overcome the drawbacks of HS, a novel BGBHS is designed for binary optimization problems. Modifications are introduced to further enhance the convergence performance. A two-phase repair operator (Algorithm \ref{alg:addAndDrop}), and a greedy selection mechanism are integrated into the BGBHS.

The initial population in BGBHS is generated randomly using a Bernoulli process. In probability and statistics, a Bernoulli process is a finite or infinite sequence of binary random variables, so it is a discrete-time stochastic process that takes only two values $\{1,0\}$, success (\ref{ec:bernoullie_p}) or failure (\ref{ec:bernoullie_p-1}), given the probability $p$. A Bernoulli process is a finite or infinite sequence of independent random variables $x_1$, $x_2$, $x_3$,\dots,$x_n$  $\forall x \in \{0,1\}$ ~ and ~ $i=\{1,\dots, n\}$.

\begin{equation} \label{ec:bernoullie_p} 
P(x_i=1) = P(\text{success at the $i$-th trial}) = p 
\end{equation}	

\begin{equation} \label{ec:bernoullie_p-1} 
P(x_i=0) = P(\text{failure at the $i$-th trial}) = 1-p
\end{equation}	

Specifically, for each decision variable of an initial harmony vector, a number within is generated randomly. If the value of the number is less than 0.5, the corresponding variable in DGHS takes 0; otherwise it takes 1. In this way, a set of HMS harmonies will be generated randomly.

The greedy operation is based on the idea that the item with higher profit density ratio should be used. And the profit density ratio can be calculated by the equation (\ref{ec:mu_j}): 

\begin{equation} \label{ec:mu_j} 
\mu_{j}={\frac{1}{c_j}}
\end{equation}	

%------------------------------Fase ADD y DROP:
\begin{algorithm}
\begin{algorithmic}[1]
 \STATE //ADD Phase
\STATE $M \gets 1,2,\ldots, m$
\STATE $A_i \sum_{j=1}^{n} a_{ij}x_{j}, i \in M$
\FOR{$j \gets 1$ \TO $n$} {
	\IF{$x_j = 0$ and $\exists i \in M, A_i < 1$ } {
		\STATE $x_j \gets 1$
		\STATE $A_i \gets A_i + a_{ij}$
	}\ENDIF
} \ENDFOR

\STATE //DROP Phase
\FOR{$j \gets n$ \TO $1$}{
	\IF{$x_j = 1$ and $\exists i \in M, A_i - a_{ij} \geq 1$ } {
		\STATE $x_j \gets 0$
		\STATE $A_i \gets A_i - a_{ij}$
	}\ENDIF
} \ENDFOR

\caption{Repair operator ADD and DROP}\label{alg:addAndDrop}
\end{algorithmic}
\end{algorithm}
%------------------------------Fase ADD y DROP:

%Owing to better performance of GHS, some modifications are introduced to further enhance the convergence performance. Then a novel binary, a two-phase repair operator \ref{alg:addAndDrop}, and a greedy selection mechanism are integrated into the BGHS, and they are described in detail as follows.

%------------------------------Diagrama de HS
\begin{figure}[H]
\centering
\begin{tikzpicture}[align = flush center, font = \small, node distance = 12mm, scale=0.6, every node/.style={scale=0.6}] %[node distance = 15mm, auto]
\node (start) [startstop] {Start};
\node (pro1) [process, below of=start] {Initialize parameters};
\node (pro2) [process, below of=pro1] {Initialize Harmony Memory (HM)};
\node (pro3) [process, below of=pro2] {Repair all harmonies in HM};
\node (pro4) [process, below of=pro3] {$t=1$};
\node (pro5) [process, below of=pro4] {Save $x_{best}$ and $x_{worst}$ harmony};
\node (proB) [process, below of=pro5]{$x_{new} = $ Bernoulli trial \smallskip $\forall {x_{new}}_j \in \{0,1\}$, and $j=\{1 \dots n\}$};
\node (pro6) [process, below of=proB] {Rand() $\leq$ HMCR};
\node (pro7) [process, below of=pro6] {Rand() $\leq$ PAR};
\node (pro8) [decision, below of=pro7] {$j < n$};
\node (pro9) [process, left of=pro8, xshift=-1.5cm] {$j = j + 1$};
\node (pro10) [decision, below of=pro8, yshift=-0.2cm] {$x_{new}$ is better than $x_{best}$};
\node (pro11) [process, right of=pro10, xshift=4.0cm] {$x_{best} = x_{new}$};
\node (pro12) [decision, below of=pro10, yshift=-1.0cm] {$x_{new}$ is better than $x_{worst}$};
\node (pro13) [process, right of=pro12, xshift=2.5cm] {$x_{worst} = x_{new}$};
\node (pro14) [decision, below of=pro12, yshift=-1.0cm] {Termination criterion is met?};
\node (pro15) [process, left of=pro14, xshift=-3.5cm] {$t=t+1$};
\node (pro16) [process, below of=pro14, yshift=-0.8cm] {Output Result};

%---------------------------------ARROWS
\draw [arrow] (start) -- (pro1);
\draw [arrow] (pro1) -- (pro2);
\draw [arrow] (pro2) -- (pro3);
\draw [arrow] (pro3) -- (pro4);
\draw [arrow] (pro4) -- (pro5);
\draw [arrow] (pro5) -- (proB);
\draw [arrow] (proB) -- (pro6);
\draw [arrow] (pro6) -- (pro7);
\draw [arrow] (pro7) -- (pro8);
\draw [arrow] (pro8) -- (pro9) node [midway, above] {Y};
\draw [arrow] (pro9) |- (pro6);
\draw [arrow] (pro8) -- (pro10) node [midway, left] {N};
\draw [arrow] (pro10) -- (pro11) node [midway, above] {Y};
\draw [arrow] (pro10) -- (pro12) node [midway, left] {N};
\draw [arrow] (pro12) -- (pro13) node [midway, above] {Y};
\draw [arrow] (pro12) -- (pro14) node [midway, left] {N};
\draw [arrow] (pro14) -- (pro15) node [midway, above] {N};
\draw [arrow] (pro15) |- (pro5);
\draw [arrow] (pro11) |- (pro14) ;
\draw [arrow] (pro13) |- (pro14) ;
\draw [arrow] (pro14) -- (pro16) ;

\end{tikzpicture}
\caption{The flowchart of BGBHS algorithm.}\label{dia:FlowBGBHS}
\end{figure}

%------------------------------nueva_armonia_agresiva():

