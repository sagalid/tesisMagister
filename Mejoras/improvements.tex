In order to test the correct execution of the metaheuristic, a total of 30 executions were performed for each instance of the benchmark, both with the $p$ parameter of Bernoulli fixed, and with the parameter $p$ variable according to the function previously shown in equation (\ref{ec:variable_p}).\\

Previously an exhaustive work was done, in order to define the most correct parameters for each instance, but on this occasion, only the tests were executed with 10 iterations until an optimal adjustment of the parameters was achieved. A particular issue is that the execution time of the tests ranged from 10 minutes to 3 hours, between instances. Given that motive, the executions were a smaller amount.\\

In the multiple experiments performed, throughout this work, it was verified that it is not convenient to modify the repair operators. For this reason this 
process was used as is. The ditail can be seen in the aforementioned biography.\\

The main improvement proposal presented in this paper has to do with the modification of the initial population generation at the beginning of metaheuristics, which aims to improve the balance between exploitation and exploration. Mainly, the exploration is improved, making the initial solution population present a much wider variation, starting with solutions where there are many activated columns $x_i = 1$ and ending with solutions that have many inactivated variables $x_i = 0$ in HM, has shown in matrix (\ref{hm_pvariable}).

\begin{align}
    	HM &= \begin{bmatrix}  1,1,1,1,1,1, \ldots,1,0,1,1\\
    			     	1,1,0,1,1,1, \ldots,1,0,1,1 \\ 
                              	1,1,0,0,1,1, \ldots, 1,0,1,1 \\
				\vdots \\
    			     	0,0,1,0,0,0, \ldots,1,0,0,0 
	\end{bmatrix}
	\label{hm_pvariable}
\end{align}

Obviously each solution must be treated by the repair operator. For the first solutions, which have a greater number of variables activated, a more intense use of DROP functionality will be made, until the solution is made feasible. On the other hand, the last solutions will have a smaller number of columns activated, so it is necessary to make intensive use of the ADD functionality of the repair operator.\\

Independent of the balance that occurs, it was demonstrated throughout the experiments that there is a greater variation in the population which has a positive impact on the convergence of the algorithm.\\

The flow of the diagram does not change with respect to the proposed flow (Figure~\ref{dia:FlowBGBHS}), since the proposed optimization focuses on generating harmonies in the initial population. It is in this instance where the parameter $ p $ for the generation of each harmony is varied.
