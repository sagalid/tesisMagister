\subsection{Metaheuristics}
Every process has a potential to be optimized. the vast majority of companies is involved in solving optimization problems. Indeed, many challenging applications in science and industry can be formulated as optimization problems.
Metaheuristics are a branch of optimization in computer science and applied mathematics
that are related to algorithms and computational complexity theory. metaheuristics are raising a large interest in diverse technologies, industries, and services since they proved to be efficient algorithms in solving a wide range of  complex real-life optimization problems in different domains.

\subsubsection{Classical Optimization Models}
Optimization problems are encountered in many domains: BigData, engineering, logistics, and business. An optimization problem may be defined by the couple $(S, f )$, where $S$ represents the set of feasible solutions, and $f : S \implies {\rm I\!R}$ the objective function to optimize. The objective function assigns to every solution
$s \in S$ of the search space a real number indicating its worth. The objective function
$f$ allows to define a total order relation between any pair of solutions in the search
space.\\

The global optimum is defined as  a solution $s^* \in S$ and it has a better objective function than all solutions of the search space, that is, $\forall  s \in  S$, $f(s^*) 	\leq f(s)$.\\

The main goal in solving an optimization problem is to find a global optimal solution $s^*$.

\begin{figure}[!h]
\centering
\begin{tikzpicture}[node distance=0.5cm][!h]
\centering
\node (pro1) [mybox] {
	\begin{minipage}{0.50\textwidth}
		 \begin{itemize}
		 	\item{Complexity and difficulty of the problem}
			\begin{itemize}
				\item{NP-completeness}
				\item{NP-completeness}
				\item{NP-completeness}
			\end{itemize}	
			\item{Requirements of the application}
		 \end{itemize}
		 
	\end{minipage}
};
\node[fancytitle, right=10pt] at (pro1.north west) {Model of the problem};

\node (pro2) [process, below of=pro1, yshift=-1.0cm] {When using metaheuristics?};
\node (pro3) [process, below of=pro2, yshift=-1.0cm] {Design of a metaheuristic};
\node (pro4) [process, below of=pro3, yshift=-1.0cm] {Implementation of the metaheuristic};
\node (pro5) [process, below of=pro4, yshift=-1.0cm] {Parameter tuning}; 


\draw [arrow] (pro1) -- (pro2);
\draw [arrow] (pro2) -- (pro3);
\draw [arrow] (pro3) -- (pro4);
\draw [arrow] (pro4) -- (pro5);

\end{tikzpicture}
\caption{Guidelines for solving SCP.}\label{fig:GuideSolvingSCP}
\end{figure}

\subsection{Main goal}
Solving the SCP using the Binary Global-Best HS metaheuristic algorithm.

\subsection{Specific objectives}
\begin{itemize}
\item Solve the SCP with the BGBHS metaheuristic and compare the results with benchmark of Beasley	\cite{citeulike:921349}	
%Solve the SCP with the HS metaheuristic and compare the results of unicost SCP using the same technique
\item Introduce new operators that make the results more close to the global optimum.
\item Search nearest optimal solutions to problems in the SCP.
\item Compare and analyze the results using different resolution strategies.
\end{itemize}




