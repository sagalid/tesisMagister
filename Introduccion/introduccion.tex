Every process is potentially optimizable. The vast majority of companies is involved in solving optimization problems every day \cite{DBLP:books/daglib/0022645}. Indeed, many challenging applications in science and industry can be modelled as optimization problems.\\

Metaheuristics are a branch of optimization in computer science that applie mathematics related to algorithms and computational complexity theory. Metaheuristics are raising a large interest in diverse technologies, industries, and services since they proved to be efficient algorithms in solving a wide range of  complex real-life optimization problems in different domains \cite{DBLP:journals/pai/Torres-JimenezP14}.\\

The instantiation of a metaheuristic requires to choose among a set of different possible components and to assign specific values to all free parameters. Instantiations are defined, by the author, as a beginning configuration, and its relevance turns crucial on solving the SCP \cite{DBLP:conf/gecco/BirattariSPV02}.\\

The SCP is a well-known mathematical problem, which tries to cover a set of needs at the lowest possible cost. SCP is very important in practice, as it has been used to model a large range of problems like scheduling, manufacturing, service planning, information retrieval, among others.\\

In an attempt to solve the SCP, a variation of Harmony Search metaheuristic in binary domain  was used. This method has been tested using SCP instances from the OR-library \cite{citeulike:921349} and this results have shown that the developed algorithm generates good solutions for each instance.

\section{Main goal}
The main goal of this work, is to solve the SCP using the Binary Global-Best Harmony Search metaheuristic algorithm. The performance of metaheuristic is validated against OR-library benchmarks resolutions.

\subsection{Specific objectives}
\begin{itemize}
\item To apply different adjustments to the parameters of initiation of the metaheuristic in experimental way to obtain better results.

\item To perform a study of the parameters driven through the experiments to see if there is a possibility to optimize by creating, eliminating or modifying some of them.

\item To repair the infeasible solutions through a repair operator.

\item To compare and to analyze the results using different resolution strategies.

\item To compare results with others metaheuristics.
\end{itemize}



\section{Related terms}

\textbf{Operator:} 
Unitary procedure for transforming information or implement the behavior of the algorithm.\\

\textbf{Solution:} 
Array of $n$ columns containing a solution for a given problem. In binary case, posible values are $0$ and $1$.\\

\textbf{Constrain:} 
Conditions to be met to find a viable solution.\\

\textbf{Benchmark:} 
Optimal set of known problem instances to validate the propose algorithm.\\

\textbf{Objective Function:}  
Implements the mathematical expression representing the problem to solve. The guiding objective function is related to the goal to achieve.\\

\textbf{Fitness:} 
Value resulting by applying the objective function to a found solution.\\

\textbf{Matrix of Costs:} 
Consist of $n$ columns vector containing the cost associated to each problem variable.\\

\textbf{Optimal Value:} 
Solution with the best fitness.\\

\textbf{Domain:} 
Set of feasible values for the variable.\\

\textbf{Matrix A:} 
Matrix containing the restrictions for the given problem.\\

\textbf{RPD:} 
Relative Percentage Deviation.\\

\textbf{Harmony Memory:} 
Memory space which includes the population of the solution vectors.\\

\textbf{Harmony Memory Size:} 
Defines the amount of harmonies that can be stored in HM.\\

\textbf{Harmony Memory Consideration Rate (HMCR):} 
In memory consideration, the value of decision variable ${x\textprime}_1$ is randomly selected from the historical values, other decision variables, $({x\textprime}_2, {x\textprime}_3,\dots,{x\textprime}_N)$ are sequentially selected in the same manner with probability where HMCR $\in$ (0,1).\\

\textbf{Pitch Adjusting Rate (PAR):} 
Each decision variable ${x\textprime}_i$ assigned a value by memory considerations is pitch adjusted with the probability of PAR where PAR $\in$ (0,1).\\

