In this chapter, a general conclusion of the work will be addressed, starting from the basis of the exhaustive detail seen in chapter \ref{sec:Analisys}.
We will then move on to a view of the overall results compared to other metaheuristics. Finally, we will make an approach that lay the foundation for future work.

\subsection{Global analysis}
This work began with the need to obtain a thorough knowledge of metaheuristics, with the ultimate aim of getting into the world of information security and to be able to apply this method in it. In order to achieve this goal the first thing was to know in depth how a metaheuristic worked, for it was studied in detail the behavior of several algorithms
To find one that met certain characteristics that made it interesting to study. In this case, Harmony Search was selected.\\

Once the algorithm or metaheuristic was selected the next step was to confront the decision to choose a type problem, to which to apply the metaheuristic and to look for possible solutions. According to what was reviewed in different publications, SCP was chosen because it had broad applications and was easy to implement.\\

After having selected Harmony Search and SCP, it is necessary to review the behavior against other metaheuristics, for which it is necessary to use a set of known data, for which we use the library of Beasley\\

After specifying the above, we evaluated the behavior of the selected metaheuristics BGBHS, compared to others with similar characteristics. As you can see in the analysis and results section. The selected metaheuristic presents poor performance in relation to Black Hole S-Shape and Improved Black Hole, according to RPD measurements.\\

Considering the comparative results of the RPDs obtained, the need to improve BGBHS was presented, for which a novel method is proposed that aims to achieve a wider variation in the initial population of solutions, thus obtaining a better convergence, but Mainly a better exploration since the solutions depart being feasible but little optimum then vary until they are feasible and close to the optimum.\\