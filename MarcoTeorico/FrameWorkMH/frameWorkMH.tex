\squeezeup
\begin{figure}[H]
\centering
\begin{tikzpicture}[]

%==============================================
%Grafico de framework de trabajo 
%==============================================
%Nodo 1==============================================
\node (pro1) [mybox] {
	\begin{minipage}{0.60\textwidth}
	\scriptsize
		 \begin{itemize}
		 	\item{Understand complexity and difficulty of the problem}
			\begin{itemize}
				\item{$NP$-completeness, size of the problem and structure.}
			\end{itemize}	
			\item{Set requirements of the application}
			\begin{itemize}
				\item{Search time, quality of solutions and robustness.}
			\end{itemize}	
		 \end{itemize}
		 
	\end{minipage}
};
\node[fancytitle, right=10pt] at (pro1.north west) {1.-Model of the problem};

%Nodo 2==============================================
\node (pro2) [mybox, below of=pro1, yshift=-2.2cm] {
	\begin{minipage}{0.70\textwidth}
		\scriptsize
		 \begin{itemize}
		 	\item{It should take into consideration the following elements in the process of designing a metaheuristic}
			\begin{itemize}
				\item{Representation}
				\item{Guiding objective function}
				\item{Constraint handling}
			\end{itemize}
			\item{Select categories of metaheuristics: Trajectory-based metaheuristics or Population-based metaheuristics}
		 \end{itemize}
		 
	\end{minipage}
};
\node[fancytitle, right=10pt] at (pro2.north west) {2.-Design of a metaheuristic};

%Nodo 3============================================
\node (pro3) [mybox, below of=pro2, yshift=-2.3cm] {
	\begin{minipage}{0.80\textwidth}
		\scriptsize
		 \begin{itemize}
		 \item{When a metaheuristic is implemented , they must be taken into cosideration the strategy that will be used in the development process}
		 \begin{itemize}
		 	\item{From scratch or no reuse (nondesirable),Code reuse,Design and code reuse (e.g., software framework ParadisEO)}
			\item{Code reuse}
			\item{Design and code reuse (e.g., software framework ParadisEO)}
		 \end{itemize}
		\end{itemize}
	\end{minipage}
};
\node[fancytitle, right=10pt] at (pro3.north west) {3.-Implementation};

%Nodo 4=============================================
\node (pro4) [mybox,below of=pro3, yshift=-2.5cm, xshift=-4.0cm] {
	\begin{minipage}{0.30\textwidth}
		\scriptsize
		 \begin{itemize}
		 	\item{Off-line: Design of experiments or meta-optimization}
		 	\item{Online: Dynamic, adaptive or self-adaptive}
		 \end{itemize}
		 
	\end{minipage}
}; 
\node[fancytitle, right=10pt] at (pro4.north west) {4.-Parameter tuning};

%Nodo 5==============================================
\node (pro5) [mybox,below of=pro3, yshift=-2.5cm, xshift=4.0cm] {
	\begin{minipage}{0.35\textwidth}
		\scriptsize
		 \begin{itemize}
		 	\item{Experimental design}
			\item{Measurements}
			\item{Reporting}
		 \end{itemize}
		 
	\end{minipage}
}; 
\node[fancytitle, right=75pt] at (pro5.north west) {5.-Performance evaluation};
%==============================================
\draw [arrow] (pro1) --  (pro2);
\draw [arrow] (pro2) -- (pro3);
\draw [arrow] (pro3) -- (pro4);
\draw [arrow] (pro3) -- (pro5);
\draw [arrow] (pro5.east) to [bend right=30] (pro1.east);
\draw [arrow] (pro5.east) to [bend right=30] (pro2.east);
\draw [arrow] (pro5.east) to [bend right=60] (pro3.east);
\end{tikzpicture}
\caption{Guidelines for solving SCP.}\label{fig:GuideSolvingSCP}
\end{figure}
\squeezeup